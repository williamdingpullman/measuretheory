\documentclass[]{book}
\usepackage{lmodern}
\usepackage{amssymb,amsmath}
\usepackage{ifxetex,ifluatex}
\usepackage{fixltx2e} % provides \textsubscript
\ifnum 0\ifxetex 1\fi\ifluatex 1\fi=0 % if pdftex
  \usepackage[T1]{fontenc}
  \usepackage[utf8]{inputenc}
\else % if luatex or xelatex
  \ifxetex
    \usepackage{mathspec}
  \else
    \usepackage{fontspec}
  \fi
  \defaultfontfeatures{Ligatures=TeX,Scale=MatchLowercase}
\fi
% use upquote if available, for straight quotes in verbatim environments
\IfFileExists{upquote.sty}{\usepackage{upquote}}{}
% use microtype if available
\IfFileExists{microtype.sty}{%
\usepackage{microtype}
\UseMicrotypeSet[protrusion]{basicmath} % disable protrusion for tt fonts
}{}
\usepackage{hyperref}
\hypersetup{unicode=true,
            pdftitle={Measure Theory},
            pdfauthor={Bill Last Updated:},
            pdfborder={0 0 0},
            breaklinks=true}
\urlstyle{same}  % don't use monospace font for urls
\usepackage{natbib}
\bibliographystyle{apalike}
\usepackage{longtable,booktabs}
\usepackage{graphicx,grffile}
\makeatletter
\def\maxwidth{\ifdim\Gin@nat@width>\linewidth\linewidth\else\Gin@nat@width\fi}
\def\maxheight{\ifdim\Gin@nat@height>\textheight\textheight\else\Gin@nat@height\fi}
\makeatother
% Scale images if necessary, so that they will not overflow the page
% margins by default, and it is still possible to overwrite the defaults
% using explicit options in \includegraphics[width, height, ...]{}
\setkeys{Gin}{width=\maxwidth,height=\maxheight,keepaspectratio}
\IfFileExists{parskip.sty}{%
\usepackage{parskip}
}{% else
\setlength{\parindent}{0pt}
\setlength{\parskip}{6pt plus 2pt minus 1pt}
}
\setlength{\emergencystretch}{3em}  % prevent overfull lines
\providecommand{\tightlist}{%
  \setlength{\itemsep}{0pt}\setlength{\parskip}{0pt}}
\setcounter{secnumdepth}{5}
% Redefines (sub)paragraphs to behave more like sections
\ifx\paragraph\undefined\else
\let\oldparagraph\paragraph
\renewcommand{\paragraph}[1]{\oldparagraph{#1}\mbox{}}
\fi
\ifx\subparagraph\undefined\else
\let\oldsubparagraph\subparagraph
\renewcommand{\subparagraph}[1]{\oldsubparagraph{#1}\mbox{}}
\fi

%%% Use protect on footnotes to avoid problems with footnotes in titles
\let\rmarkdownfootnote\footnote%
\def\footnote{\protect\rmarkdownfootnote}

%%% Change title format to be more compact
\usepackage{titling}

% Create subtitle command for use in maketitle
\providecommand{\subtitle}[1]{
  \posttitle{
    \begin{center}\large#1\end{center}
    }
}

\setlength{\droptitle}{-2em}

  \title{Measure Theory}
    \pretitle{\vspace{\droptitle}\centering\huge}
  \posttitle{\par}
    \author{Bill Last Updated:}
    \preauthor{\centering\large\emph}
  \postauthor{\par}
      \predate{\centering\large\emph}
  \postdate{\par}
    \date{27 January, 2020}

\usepackage{booktabs}
\usepackage{amsthm}
\makeatletter
\def\thm@space@setup{%
  \thm@preskip=8pt plus 2pt minus 4pt
  \thm@postskip=\thm@preskip
}
\makeatother

\begin{document}
\maketitle

{
\setcounter{tocdepth}{1}
\tableofcontents
}
\hypertarget{measure-spaces}{%
\chapter{Measure Spaces}\label{measure-spaces}}

\hypertarget{sigma-fields}{%
\section{\texorpdfstring{\(\sigma\)-Fields}{\textbackslash{}sigma-Fields}}\label{sigma-fields}}

Let \(\mathcal{F}\) be a collection of subsets of a set \(\Omega\). Then, \(F\) is called a sigma field (or, sigma algebra; written as \(\sigma\)-field or \(\sigma\)-algebra) if and only if it satisfies the following properties:

\begin{enumerate}
\def\labelenumi{(\roman{enumi})}
\item
  The empty set \(\emptyset \in \mathcal{F}\).
\item
  If \(A \in \mathcal{F}\), then the complement \(A^c \in \mathcal{F}\).
\item
  If \(A_1, A_2,...\) is a sof elements of \(\mathcal{F}\), then their union \(\cup_{i=1}^\infty A_i \in \mathcal{F}\).
\end{enumerate}

A pair (\(\Omega, \mathcal{F}\)) consisting of a set \(\Omega\) and a \(\sigma-field\) of subsets \(\mathcal{F}\) is called a measurable space. The elements of \(\mathcal{F}\) are called measurable sets or events.

\textbf{Remarks}

\begin{enumerate}
\def\labelenumi{(\arabic{enumi})}
\item
  The set \(\Omega\) is called sample space in probability, but in general measure theory it is called the underlying set or underlying space.
\item
  Since \(\emptyset^c=\Omega\), it follows (i) and (ii) \(\Omega \in \mathcal{F}\).
\item
  Given any set \(\Omega\), the trivial \(\sigma\)-field is \(\mathcal{F}=\{ \emptyset, \Omega\}\). One can easily verify that this is a \(\sigma\)-field, and is in fact the smallest \(\sigma\)-field on \(\Omega\).
\item
  Given any set \(\Omega\), the power set:
\end{enumerate}

\[\mathcal{P}(\Omega)=\{A: A \in \Omega\}\]

Consisting of all subsets of \(\Omega\) is also a \(\sigma\)-field on \(\Omega\)

\bibliography{book.bib,packages.bib}


\end{document}
