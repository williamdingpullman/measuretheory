\documentclass[]{book}
\usepackage{lmodern}
\usepackage{amssymb,amsmath}
\usepackage{ifxetex,ifluatex}
\usepackage{fixltx2e} % provides \textsubscript
\ifnum 0\ifxetex 1\fi\ifluatex 1\fi=0 % if pdftex
  \usepackage[T1]{fontenc}
  \usepackage[utf8]{inputenc}
\else % if luatex or xelatex
  \ifxetex
    \usepackage{mathspec}
  \else
    \usepackage{fontspec}
  \fi
  \defaultfontfeatures{Ligatures=TeX,Scale=MatchLowercase}
\fi
% use upquote if available, for straight quotes in verbatim environments
\IfFileExists{upquote.sty}{\usepackage{upquote}}{}
% use microtype if available
\IfFileExists{microtype.sty}{%
\usepackage{microtype}
\UseMicrotypeSet[protrusion]{basicmath} % disable protrusion for tt fonts
}{}
\usepackage{hyperref}
\hypersetup{unicode=true,
            pdftitle={Measure Theory},
            pdfauthor={Bill Last Updated:},
            pdfborder={0 0 0},
            breaklinks=true}
\urlstyle{same}  % don't use monospace font for urls
\usepackage{natbib}
\bibliographystyle{apalike}
\usepackage{longtable,booktabs}
\usepackage{graphicx,grffile}
\makeatletter
\def\maxwidth{\ifdim\Gin@nat@width>\linewidth\linewidth\else\Gin@nat@width\fi}
\def\maxheight{\ifdim\Gin@nat@height>\textheight\textheight\else\Gin@nat@height\fi}
\makeatother
% Scale images if necessary, so that they will not overflow the page
% margins by default, and it is still possible to overwrite the defaults
% using explicit options in \includegraphics[width, height, ...]{}
\setkeys{Gin}{width=\maxwidth,height=\maxheight,keepaspectratio}
\IfFileExists{parskip.sty}{%
\usepackage{parskip}
}{% else
\setlength{\parindent}{0pt}
\setlength{\parskip}{6pt plus 2pt minus 1pt}
}
\setlength{\emergencystretch}{3em}  % prevent overfull lines
\providecommand{\tightlist}{%
  \setlength{\itemsep}{0pt}\setlength{\parskip}{0pt}}
\setcounter{secnumdepth}{5}
% Redefines (sub)paragraphs to behave more like sections
\ifx\paragraph\undefined\else
\let\oldparagraph\paragraph
\renewcommand{\paragraph}[1]{\oldparagraph{#1}\mbox{}}
\fi
\ifx\subparagraph\undefined\else
\let\oldsubparagraph\subparagraph
\renewcommand{\subparagraph}[1]{\oldsubparagraph{#1}\mbox{}}
\fi

%%% Use protect on footnotes to avoid problems with footnotes in titles
\let\rmarkdownfootnote\footnote%
\def\footnote{\protect\rmarkdownfootnote}

%%% Change title format to be more compact
\usepackage{titling}

% Create subtitle command for use in maketitle
\providecommand{\subtitle}[1]{
  \posttitle{
    \begin{center}\large#1\end{center}
    }
}

\setlength{\droptitle}{-2em}

  \title{Measure Theory}
    \pretitle{\vspace{\droptitle}\centering\huge}
  \posttitle{\par}
    \author{Bill Last Updated:}
    \preauthor{\centering\large\emph}
  \postauthor{\par}
      \predate{\centering\large\emph}
  \postdate{\par}
    \date{27 January, 2020}

\usepackage{booktabs}
\usepackage{amsthm}
\makeatletter
\def\thm@space@setup{%
  \thm@preskip=8pt plus 2pt minus 4pt
  \thm@postskip=\thm@preskip
}
\makeatother

\begin{document}
\maketitle

{
\setcounter{tocdepth}{1}
\tableofcontents
}
\hypertarget{my-section}{%
\chapter*{Motivation}\label{my-section}}
\addcontentsline{toc}{chapter}{Motivation}

Based on Dr.~Dennis Cox's ``The Theory of Statistics and Its Applications,'' I made the note for personal study purpose using RMarkdown and being posted in Github. All copyright belongs to Dr.~Cox, and thus please do not copy, download, or distribute under any circumstance.

\hypertarget{measure-spaces}{%
\chapter{Measure Spaces}\label{measure-spaces}}

\hypertarget{measure}{%
\section{Measure}\label{measure}}

A measure is a function \(\mu\) defined for certain subsets \(A\) of a set \(\Omega\) which assigns a nonnegative number \(\mu(A)\) to each ``measurable'' set \(A\). In probability theory, the probability is a measure, denoted by \(P\) instead of \(\mu\), which satisfies \(P(\Omega)=1\). In the context of probablity theory, the subset \(A\) is called an event, and \(\Omega\) is called the sample space.

\textbf{Example}

\begin{enumerate}
\def\labelenumi{(\arabic{enumi})}
\item
  A die with 6 faces. Thus, the sample space \(\Omega\) is the finite set of integers \(\{1, 2, 3, 4, 5, 6\}\) corresponding to the possible outcomes if we roll the die once and count the number of spots on the fact that turns up. We define a probablity measure \(P\) hese events by \(P(A)=Count(A)/6\).
\item
  Let a random number be chosen in the interval \([0,1]\) such that the probability of the number lying in any subinterval \([a,b] (0 \leq a < b \leq 1)\) is the length (i.ee., \(P([a,b])=b-a\)). Thus, here \(\Omega=[0,1]\). Such a random number is said to be uniformly distributed on the interval \([0,1]\). We can extend the probability measure \(P\) from closed intervals to other subsets of \([0,1]\) (e.g., \(P((a,b))=P([a,b))=P((a,b])=P([a,b])=b-a\)). Also, if \([a_1,b_1],[a_2,b_2],...\) is a finite or infinite sequence of disjoint closed intervals (one can also allow open or semi-open intervals), then we can get \(P(\cup_i[a_i,b_i])=\sum_i(b_i-a_i)\).(It turns out for technical reasons that in this case, one cannot define probability measure of all subsets of \([0,1]\).)
\end{enumerate}

The probability measure of this example is related to a (nonprobability) measure: Let \(\mu=m\) be a measure on arbitrary intervals of real numbers which equals the length of the interval, i.e., \(m((a,b))=b-a\) for any open interval \((a,b),a<b\), and similarly for the other varieties of intervals. Here, \(\Omega=\mathbb{R}\), the set of all real numbers, also denoted \((-\infty, \infty)\). This measure \(m\) is called Lebesgue measure.

\hypertarget{sigma-fields}{%
\subsection{\texorpdfstring{\(\sigma\)-Fields}{\textbackslash{}sigma-Fields}}\label{sigma-fields}}

\textbf{Definition 1}

Let \(\mathcal{F}\) be a collection of subsets of a set \(\Omega\). Then, \(F\) is called a sigma field (or, sigma algebra; written as \(\sigma\)-field or \(\sigma\)-algebra) if and only if it satisfies the following properties:

\begin{enumerate}
\def\labelenumi{(\roman{enumi})}
\item
  The empty set \(\emptyset \in \mathcal{F}\).
\item
  If \(A \in \mathcal{F}\), then the complement \(A^c \in \mathcal{F}\).
\item
  If \(A_1, A_2,...\) is a sof elements of \(\mathcal{F}\), then their union \(\cup_{i=1}^\infty A_i \in \mathcal{F}\).
\end{enumerate}

A pair (\(\Omega, \mathcal{F}\)) consisting of a set \(\Omega\) and a \(\sigma-field\) of subsets \(\mathcal{F}\) is called a measurable space. The elements of \(\mathcal{F}\) are called measurable sets or events.

\textbf{Remarks 1}

\begin{enumerate}
\def\labelenumi{(\arabic{enumi})}
\item
  The set \(\Omega\) is called sample space in probability, but in general measure theory it is called the underlying set or underlying space.
\item
  Since \(\emptyset^c=\Omega\), it follows (i) and (ii) \(\Omega \in \mathcal{F}\).
\item
  Given any set \(\Omega\), the trivial \(\sigma\)-field is \(\mathcal{F}=\{ \emptyset, \Omega\}\). One can easily verify that this is a \(\sigma\)-field, and is in fact the smallest \(\sigma\)-field on \(\Omega\).
\item
  Given any set \(\Omega\), the power set:
\end{enumerate}

\[\mathcal{P}(\Omega)=\{A: A \in \Omega\}\]

Consisting of all subsets of \(\Omega\) is also a \(\sigma\)-field on \(\Omega\), and in fact is the largest \(\sigma\)-field on \(\Omega\) (Note: in some text, people denote \(\mathcal{P}(\Omega)\) by \(2^\Omega\)).

\begin{enumerate}
\def\labelenumi{(\arabic{enumi})}
\setcounter{enumi}{4}
\tightlist
\item
  It follows from the definition that if \(\mathcal{F}\) is a \(\sigma\)-field and \(A_1, A_1,...\) is a sequence in \(\mathcal{F}\), then the intersection \(\cap_{i=1}^\infty A_i \in \mathcal{F}\).
\end{enumerate}

\textbf{Remarks 2}

\begin{enumerate}
\def\labelenumi{(\arabic{enumi})}
\tightlist
\item
  A set \(A\) is called countable it can be listed as a sequence, finite, or infinite:
\end{enumerate}

\[A=\{a_1, a_2,...\}\]

We shall sometimes say that a set is countably infito indicate that it is countable but not finite.

\begin{enumerate}
\def\labelenumi{(\arabic{enumi})}
\setcounter{enumi}{1}
\item
  Any set which can be put in one to one correspondence with a subset of the natural numbers or ``counting numbers'' \(\mathcal{N}={1,2,3...}\) is called countable.
\item
  The set of all real numbers \(\mathbb{R}\) (including irrational numbers like \(\sqrt{2}\), \(\pi\), \(e\)) can not put into a one to one correspondence with the natural numbers, so it has ``more'' elements, and is said to be uncountably infinite. These issues are pertinent to the technical difficulties wich make it impossible to extend Lebesgue measure to all subsets of \(\mathbb{R}\), and hence require us to consider the notion of a \(\sigma\)-field (rather than just defining a measure on all subsets of the underlying space).
\end{enumerate}

It takes a certain amount of work to obtain \(\sigma\)-field other than the trivial \(\sigma\)-field (i.e., \(\{\emptyset, \Omega\}\)) or the power set (i.e., \(\mathcal{P}(\Omega)=\{A: A \in \Omega\}\)). A standard approach is to consider the smallest \(\sigma\)-field containing a given family of sets. We shall illusrtate this concept. Let \(A \in \Omega\) be a nonempty proper subset of \(\Omega\) (i.e., \(\emptyset \neq A \neq \Omega\)), then

\[\sigma(A)=\{\emptyset, A, A^c, \Omega\}\]

is a \(\sigma\)-algebra (or, \(\sigma\)-field).

\textbf{Example}

If \(\emptyset \neq B \neq \Omega\), \(A\cap B \neq \emptyset\), and neither \(A\) or \(B\) is a subset of the other, then one can obtain a \(\sigma\)-fied \(\sigma(\{A, B\})\) consisting of 16 element (i.e., \(2^4=16\)).

\bibliography{book.bib,packages.bib}


\end{document}
